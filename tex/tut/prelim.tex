
\title{How to Read}
%\author{}
\date{}

\begin{document}

\section{Shell}
The commands in my tutorials assume a POSIX-compliant shell.
So if your default shell is \ilname{tcsh}, then please type
\begin{code}
bash
\end{code}
before executing any commands.

\section{Variables}
I often use shell variables instead of dummy names in command examples so that someone can set those variables and then copy/paste the commands.
(It's not great practice to do this from untrusted sources on the internet, but there's your \textbf{disclaimer}.)
So, if my guide says to assume \ilcode{user=my_awesome_username}, and your username is ``leethaxor'', then you should run
\begin{code}
user=leethaxor
\end{code}
before running the commands after that.
Those commands will substitute occurrences of \ilcode{${user}} with \ilcode{leethaxor}.

I often write \ilcode{${user}} or even \ilcode{"${user}"} for clarity/safety because many commands are meant to be copy/pasted.
When typing it yourself, it's usually fine to just use \ilcode{$user} if the string does not contain spaces (e.g., a username should not).

\end{document}

