
\newcommand{\AKA}{a.k.a.}

\begin{document}

A distributed system (\AKA protocol) normally provides some service within a set of legitimate states.
A \textit{self-stabilizing} system converges to this intended behavior from any state, even from states where all data is randomized by transient faults.
My PhD work shows that automatically designing (\AKA synthesizing) stabilization is NP-complete, and verifying stabilization is undecidable when the protocol must scale to arbitrarily many processes.
Even so, I had some success in designing protocols using a backtracking synthesis algorithm.
These results are all in my \href{limpract.pdf}{dissertation}, and the synthesis algorithm is implemented in an open-source tool called
\href{https://github.com/grencez/protocon}{Protocon} (see \href{http://grencez.github.io/protocon}{its webpage} for more info).

My focus thus far has been on protocols with very simple processes.
This is a necessity for Protocon's synthesis algorithm, since it can only realistically find a solution when process domains are small.
In this work, I have grown to appreciate \href{https://en.wikipedia.org/wiki/Emergence}{emergent} properties of a system, where the simple components perform some more complicated task.
We can view legitimate states/behavior as emergent properties of a self-stabilizing system, which relies on simple processes to create order from chaos (arbitrary initial state).
This view assumes processes are simple, which is the case for many good papers on self-stabilizing protocols because simple processes make analysis easier (yet it is still quite hard).
With this in mind, I will focus more on protocols that can apply to \href{https://en.wikipedia.org/wiki/Self-reconfiguring_modular_robot}{modular robots}, as they will inevitably get smaller and become resource-constrained.

\end{document}
